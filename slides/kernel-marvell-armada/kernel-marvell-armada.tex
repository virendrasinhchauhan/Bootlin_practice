\section{Marvell EBU mainlining}

\begin{frame}{Marvell EBU mainlining}

\begin{itemize}
\item Collaboration between Marvell and Free Electrons
\item Goal is to integrate in the upstream Linux kernel the support
  for Marvell EBU processors
\item Started in April 2012, with the Armada 370/XP
\item Extended to cover Armada 375/38x in the second half of 2013
\item On-going project
\item On Free Electrons side:
  \begin{itemize}
  \item Grégory Clement
  \item Ezequiel Garcia
  \item Thomas Petazzoni
  \end{itemize}
\end{itemize}

\end{frame}

\begin{frame}{Process}
  \begin{center}
    \includegraphics[width=\textwidth]{slides/kernel-marvell-armada/process.pdf}
  \end{center}
\end{frame}

\begin{frame}{Submission}
  \begin{center}
    \includegraphics[width=\textwidth]{slides/kernel-marvell-armada/submission.pdf}
  \end{center}
\end{frame}

\begin{frame}{Timeline (1)}
  \begin{center}
    \includegraphics[width=\textwidth]{slides/kernel-marvell-armada/timeline1.pdf}
  \end{center}
\end{frame}

\begin{frame}{Timeline (2)}
  \begin{center}
    \includegraphics[width=\textwidth]{slides/kernel-marvell-armada/timeline2.pdf}
  \end{center}
\end{frame}

\begin{frame}{Using the result}

\begin{itemize}
\item \code{make mvebu_defconfig} (to support just mvebu platforms) or
  \code{make multi_v7_defconfig} (for the ARMv7 multiplatform kernel)
\item \code{make}
  \begin{itemize}
  \item Kernel image: \code{arch/arm/boot/zImage}
  \item Device Tree blobs: \code{arch/arm/boot/dts/armada-*.dtb}
  \end{itemize}
\item From U-Boot
  \begin{itemize}
  \item If DT capable: \code{bootz <kerneladdr> - <dtbaddr>}
  \item If not DT capable
    \begin{itemize}
    \item Kernel must have \code{CONFIG_ARM_APPENDED_DTB=y}, and
      optionally \code{CONFIG_ARM_ATAG_DTB_COMPAT=y}.
    \item \code{cat zImage <dtb> > zImage.<foo>}
    \item \code{mkimage -A arm -O linux -C none -a <loadaddr> -e <entrypoint> -d zImage.<foo> uImage.<foo>}
    \item \code{bootm <kerneladdr>}
    \end{itemize}
  \end{itemize}
\end{itemize}

\end{frame}

\begin{frame}{Where is the code: platform code (1)}

\begin{itemize}
\item One single directory for all Marvell EBU platforms:
  \code{arch/arm/mach-mvebu/}
\item Currently has Armada 370, XP, 375, 38x, but also Dove and
  Kirkwood.
\item Will ultimately also have Orion5x support, and possibly MV78xx0.
\item Goal is to remove \code{mach-dove}, \code{mach-orion5x},
  \code{mach-mv78xx0} and \code{plat-orion}
\item Platform code should be as minimal as possible: most of the code
  should go in drivers.
\end{itemize}

\end{frame}

\begin{frame}{Where is the code: platform code (2)}

\footnotesize
\begin{columns}
\column{0.5\textwidth}
\begin{itemize}
\item {\bf board-v7.c} Core file for v7 platforms, includes the
  \code{DT_MACHINE_START} definition.
\item {\bf coherency.c} Coherency fabric code (for CPU coherency on
  A370/XP and I/O coherency 370, XP, 375 and 38x)
\item {\bf cpu-reset.c} Minimal driver to handle the CPU reset
  register, used for SMP boot.
\item {\bf dove.c} Core file for Dove platforms. Should be merged with
  \code{board-v7.c} in the future.
\item {\bf kirkwood.c} Core file for Kirkwood platforms. Should be
  renamed \code{board-v5.c} in the future.
\item {\bf kirkwood-pm.c} Suspend/resume support for Kirkwood
\end{itemize}
\column{0.5\textwidth}
\begin{itemize}
\item {\bf mvebu-soc-id.c} SoC identification for 370, XP, 375, 38x
\item {\bf netxbig.c} Remaining board support for Kirkwood. Should be
  removed at some point.
\item {\bf platsmp.c, platsmp-a9.c, headsmp.S, headsmp-a9.S} SMP
  support for Armada 370/XP and Armada 375/38x
\item {\bf pmsu.c} Power Management Service Unit of Armada 370, XP,
  38x (used for SMP, cpuidle, cpufreq, etc.)
\item {\bf system-controller.c} Used for platform reset, SMP boot on
  some platforms.
\end{itemize}
\end{columns}

\end{frame}

\begin{frame}{Where is the code: Device Tree}

In \code{arch/arm/boot/dts}
\footnotesize
\begin{columns}
\column{0.5\textwidth}
\begin{itemize}
\item armada-370.dtsi
\item armada-370-xp.dtsi
\item armada-375.dtsi
\item armada-380.dtsi
\item armada-385.dtsi
\item armada-38x.dtsi
\item armada-xp.dtsi
\item armada-xp-mv78230.dtsi
\item armada-xp-mv78260.dtsi
\item armada-xp-mv78460.dtsi
\end{itemize}
\column{0.5\textwidth}
\begin{itemize}
\item armada-370-db.dts
\item armada-370-mirabox.dts
\item armada-370-netgear-rn102.dts
\item armada-370-netgear-rn104.dts
\item armada-370-rd.dts
\item armada-375-db.dts
\item armada-385-db.dts
\item armada-385-rd.dts
\item armada-xp-axpwifiap.dts
\item armada-xp-db.dts
\item armada-xp-gp.dts
\item armada-xp-matrix.dts
\item armada-xp-netgear-rn2120.dts
\item armada-xp-openblocks-ax3-4.dts
\end{itemize}
\end{columns}

\end{frame}

\begin{frame}{Where is the code: drivers (1)}

\begin{itemize}
\item Clock drivers: \code{drivers/clk/mvebu}
\item L2 cache: \code{arch/arm/mm/cache-l2x0.c} (for both Aurora and PL310)
\item MPIC: \code{drivers/irqchip/irq-armada-370-xp.c}, includes MSI
  and IPI support
\item GIC: \code{drivers/irqchip/irq-gic.c}
\item Marvell timer: \code{drivers/clocksource/time-armada-370-xp.c}
\item ARM timer: \code{arch/arm/kernel/smp_twd.c}
\item Pin-muxing drivers (MPPs): \code{drivers/pinctrl/mvebu}
\item GPIO driver: \code{drivers/gpio/gpio-mvebu.c}
\item Watchdog: \code{drivers/watchdog/orion_wdt.c}
\end{itemize}

\end{frame}

\begin{frame}{Where is the code: drivers (2)}

\begin{itemize}
\item MBus: \code{drivers/bus/mvebu-mbus.c}, configuration of CPU to
  device windows, provide an API for drivers to query SDRAM windows
\item Device Bus: \code{drivers/memory/mvebu-devbus.c}, for NOR
  support mainly
\item PCIe: \code{drivers/pci/host/pci-mvebu.c}. Includes a PCI-to-PCI
  bridge emulation for dynamic MBus window allocation.
\item Network 370/XP/38x: \code{drivers/net/ethernet/marvell/mvneta.c}
\item Network 375: \code{drivers/net/ethernet/marvell/mvpp2.c}
\end{itemize}

\end{frame}

\begin{frame}{Where is the code: drivers (3)}

\begin{itemize}
\item SATA 370/XP/375: \code{drivers/ata/sata_mv.c} (existing driver,
  shared with Kirkwood, etc.)
\item SATA 38x: \code{drivers/ata/ahci_mvebu.c}
\item RTC, \code{drivers/rtc/rtc-mv.c} (existing driver, used on Kirkwood)
\item SPI, \code{drivers/spi/spi-orion.c} (existing driver, used on Kirkwood)
\item I2C, \code{drivers/i2c/busses/i2c-mv64xxx.c} (existing driver,
  extended to support the transaction generator)
\item Serial, \code{drivers/tty/serial/8250/}
\end{itemize}

\end{frame}

\begin{frame}{Where is the code: drivers (4)}

\begin{itemize}
\item USB, \code{drivers/usb/host/ehci-orion.c} (existing driver, used
  on Kirkwood), \code{drivers/usb/host/xhci-plat.c} (for XHCI)
\item NAND, \code{drivers/mtd/nand/pxa3xx_nand.c} (existing driver,
  but significantly extended to support large pages)
\item SDIO, \code{drivers/mmc/host/mvsdio.c} (existing driver, used on
  Kirkwood)
\item Thermal, \code{drivers/thermal/armada_thermal.c}, new driver
\item XOR, \code{drivers/dma/mv_xor.c} (existing driver, used on
  Kirkwood)
\end{itemize}

\end{frame}

\begin{frame}{Programming practices}
  \begin{itemize}
  \item Device Tree
    \begin{itemize}
    \item Problem of stability: once accepted, a Device Tree binding
      cannot be changed.
    \item Need to identify {\em hardware blocks}. For some core SoC
      features, lack of good separation between hardware blocks, difficult
      to create nodes in the Device Tree.
    \end{itemize}
  \item Multiplatform
    \begin{itemize}
    \item One single kernel image for all v6/v7 platforms
    \item Everything must be runtime detected (from the Device Tree or by
      looking at the hardware)
    \item {\bf No more compile time conditionals!}
    \end{itemize}
  \end{itemize}
\end{frame}